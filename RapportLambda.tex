\hypertarget{perkin-elmer-lambda-9}{%
\section{Perkin Elmer Lambda 9}\label{perkin-elmer-lambda-9}}

\hypertarget{probluxe8me}{%
\subsection{Problème}\label{probluxe8me}}

Actuellement le computer unit effectue la mesure et l'imprime au fur et
à mesure. Il n'est pas possible de sauvegarder les données sur un
ordinateur.\\
Le protocol de communication avec l'imprimante est inconnu.

\hypertarget{choses-que-jai-essayuxe9}{%
\subsection{Choses que j'ai essayé:}\label{choses-que-jai-essayuxe9}}

Brancher l'imprimante sur un ordinateur avec un convertissuer usb
parralel pour voir si elle est reconnue. non elle ne l'est pas\\
essayé d'imprimer un fichier texte avec l'imprimante Printer/Plotter. ça
ne marche pas meme en essayant plusieurs drivers différents\\
Lu le manuel, rien trouvé sur la communication avec l'imprimante ni sur
une sortie de donnée\\
Chercher des informations sur l'imprimante Printer/Plotter et son mode
de communication sur internet. Rien trouvé.\\
Chercher des informations sur comment sortir des informations de mesure
de l'appareil.\\
Essayé de connecter l'ordinateur sur le port série du Lambda 9 et de
suivre \href{https://ftgsoftware.com/issues_pe_ir_com.htm}{ces
instructions} mais aucune réponse\\
Trouvé un site
\href{https://ftgsoftware.com/issues_lambda19.htm}{FtgSoftware} qui
donne quelques informations éparses sur un logiciel \texttt{Lambda\ SPX}
avec des prérequis très précis et un convertisseur 2*sub D25 vers sub
D37 qu'il faut fabriquer soi meme.\\
Trouvé des appareils de capture de sortie imprimante:

\sout{\href{https://www.retroprinter.com/}{Retro Printer} qui permet de
capturer la sortie imprimante et de la sauvegarder sur un ordinateur.
Vendu directement.}\\
\sout{\href{https://github.com/bkw777/LPT_Capture}{Lpt Capture} idem
mais plus petit et pas vendu directement}\\
\sout{\href{https://tomverbeure.github.io/2023/01/24/Fake-Parallel-Printer-Capture-Tool-HW.html}{Fake
parralel printer capture} similaire à LPT Capture.}\\
(Ce n'est pas un port parallele mais un port série que l'on essaie de
capturer)

Malheureusement je ne sais pas si ces appareils fonctionneront avec le
Lambda 9 car la communication avec l'imprimante est inconnue. Je pense
qu'il est possible de capturer les données si l'ordinateur du Lambda 9
utilise les signaux standarts pour transmettre de l'information à
l'imprimante.

\hypertarget{protocol-analyzer}{%
\subsubsection{Protocol analyzer}\label{protocol-analyzer}}

Auto config du protocol analyzer:

protocol: Char Async/sync\\
bit order: LSB 1st\\
code: Ascii 8\\
Parity: none\\
transpar: none\\
Mode: async 1\\
bits/Sec: 9600\\
disp mode: D \& S\\
supress: none

extrait d'une trame de communication:

\begin{verbatim}
IT,Z0,F15936,416,0,200,D0218,1280,A1,X2200,-100,5,S2100.0,D1,1,Y120.0,-14.000,4,Z0,D0128,1280,L1\r\r  
\end{verbatim}

\begin{verbatim}
A0,D1,1,M0,10\r  
\end{verbatim}

\hypertarget{programme-simulate-dce}{%
\subsubsection{Programme Simulate DCE}\label{programme-simulate-dce}}

Programme qui simule l'imprimante sur le HP4951:

\begin{verbatim}
Simulate DCE  
  
Block 1  
Set Lead DSR On  
    and then  
Set Lead CD On  
    and then  
Set Lead CTS On  
    and then  
Send F,00\r  
  
Block 2  
When DTE \r  
    then goto Block 3  
  
Block 3  
Send 01\r  
    and then  
Goto Block 2  
\end{verbatim}

\hypertarget{informations}{%
\subsubsection{Informations}\label{informations}}

La plage des mesures est sur 14 bits(16384)\\
Malgrès le db25, la communication se fait en série (en désactivant
certains pin la communication fonctionne toujours) Inversement de
certaines lignes quand on connecte l'ordinateur au Lambda 9

\hypertarget{cablage}{%
\subsubsection{Cablage}\label{cablage}}

Cablage qui fonctionne pour brancher l'ordinateur au lambda 9:\\
PC -\textgreater{} Cable DB9 null Modem -\textgreater{} convertisseur
DB9 vers DB25 -\textgreater{} raccordement des pins 4,5,6,8
(RTS,CTS,DSR,DCD) ensemble -\textgreater{} Cable DB25 -\textgreater{}
Lambda 9

\hypertarget{cablage-duxe9finitif}{%
\subsubsection{Cablage définitif:}\label{cablage-duxe9finitif}}

{[}{[}./images/cableLambda.svg{]}{]}
